\documentclass[UTF8,10pt,a4paper]{ctexart}
\usepackage[top=1in, bottom=1in, left=1in, right=1in]{geometry}


%画图包
%\usepackage{tikz}
\usepackage{tikz,tkz-tab}
\usetikzlibrary{shapes,arrows,backgrounds}
\usepackage{amsmath,bm,times}
\usepackage{xcolor} %代码高亮


%表格
\usepackage{tabularx,colortbl}
\newcommand{\gray}{\rowcolor[gray]{.90}} % Custom highlighting for the work experience and education sections

%字体
\usepackage{fontspec,xltxtra,xunicode}
%\defaultfontfeatures{Mapping=tex-text}
\setmonofont[Mapping={}]{Times New Roman} %让导入的代码引号直立
%\setmonofont{Consolas}
%\setromanfont{STZhongsong}
%\setmainfont{Times New Roman} 

%section
\ctexset{section/format=\Large\bfseries}

%图片导入
\usepackage{graphicx}
\usepackage{subfigure}
\usepackage{float}
\graphicspath{{./figs/}{./}{./figs2/}}

%logo & 页眉
\usepackage{fancyhdr}
\pagestyle{fancy}
\lhead{\includegraphics[scale=0.1]{BGIlogo.png}}

%附录                                                                                                                                                                                
\usepackage[titletoc]{appendix}

%超链接
%\usepackage[hyperfootnotes=true]{hyperref}
\usepackage[colorlinks,
            linkcolor=red,
            anchorcolor=blue,
            citecolor=green,
            urlcolor=blue]{hyperref}


%原文照列
\usepackage{verbatim}
\usepackage{clrscode}
\usepackage{listings} %插入代码
 
\lstset{numbers=left, %设置行号位置
	backgroundcolor=\color[RGB]{245,245,244},
	basicstyle=\footnotesize,
	showstringspaces=false,
        numberstyle=\tiny, %设置行号大小
        keywordstyle=\color{blue}, %设置关键字颜色
        commentstyle=\color[cmyk]{1,0,1,0}, %设置注释颜色
        frame=single, %设置边框格式
   %     escapeinside=``, %逃逸字符(1左面的键),用于显示中文
       basicstyle=\scriptsize\ttfamily,
        breaklines, %自动折行
        extendedchars=false, %解决代码跨页时,章节标题,页眉等汉字不显示的问题
        xleftmargin=2em,xrightmargin=2em, aboveskip=1em, %设置边距
        tabsize=4, %设置tab空格数
        showspaces=false, %不显示空格
        %belowcaptionskip=1\baselineskip,
        stringstyle=\color{orange}
        }

\title{Gaea Pipeline V2.0说明文档{\small (V1.3)}}
%\author{ \href{mailto:huangzhibo@genomics.cn}{黄志博}}
\author{\href{http://bigdata.genomics.cn/}{大数据计算组} \and 
\href{mailto:huangzhibo@genomics.cn}{黄志博}}
\date{\today}

\begin{document}
\maketitle
%\vspace{2em}
\tableofcontents
\newpage
\setlength{\parskip}{1ex plus 0.5ex minus 0.2ex}

%定义方框形状
\tikzstyle{dir}= [draw=black!50, fill=blue!15, text width=7em,  text centered, minimum height=2.8em, preaction={fill=black,opacity=.4,
transform canvas={xshift=0.7mm,yshift=-0.7mm}}]
\tikzstyle{app}= [draw, fill=red!20, text width=5em,  text centered, minimum height=2.3em, preaction={fill=black,opacity=.4,
transform canvas={xshift=0.7mm,yshift=-0.7mm}}]
\tikzstyle{module}= [draw, fill=red!20, text width=8em,  text centered, minimum height=3em, preaction={fill=black,opacity=.4,
transform canvas={xshift=0.7mm,yshift=-0.7mm}}]
\tikzstyle{head}=[rectangle,rounded corners,text centered,fill=blue!50,draw=black!50, text width=18em,minimum height=3em,
preaction={fill=black,opacity=.4,
transform canvas={xshift=0.7mm,yshift=-0.7mm}}]
\tikzstyle{bg}=[rectangle,fill=gray!60,rounded corners, draw=black!50, text width=18em,minimum height=19em]
\tikzstyle{bgbox}=[rectangle,fill=blue!15,rounded corners, draw=black!50, dashed,text width=6.2em,minimum height=12em]
\tikzstyle{blankbox}=[rectangle,fill=blue!15,rounded corners,text width=6.2em,minimum height=10em]

%其他形状
\tikzstyle{io} = [trapezium, trapezium left angle=70, trapezium right angle=110, text width=1.1cm, minimum height=1cm, text centered, draw=blue, fill=blue!30,
preaction={fill=black,opacity=.4,
transform canvas={xshift=0.7mm,yshift=-0.7mm}}]

\section{概述}
\label{sec:1}


\subsection{新版本说明}
随着Gaea平台的应用越来越多,原流程(Gaea\_Pipeline\_standard\_output\_multi\_beta.pl)变得越来越复杂,为了便于日后的流程维护和应用扩展,开发了此流程(python),定名为GaeaPipelineV2.0.0 beta。新版本有如下特征:
\begin{itemize}
\item 采用模块化设计,便于APP的扩展,若增加新的分析模块只需按约定写好APP并存放到相应目录即可。
\item hadoop任务与单机任务之间能更好的衔接,hadoop任务完成后单机任务会分样本投递任务,并分别考虑单个样本内任务依赖关系。
\item 优化了用户参数配置文件,采用cfg文件格式,并简化参数配置,使参数配置更加简单。
\item 优化了Hadoop任务运行错误检测方法,根据任务脚本标准错误输出信息的更多指标对任务执行状态进行判断。
\item 使用新的任务重投方式,单样本分析时能更灵活的选择重跑步骤。
\end{itemize}

\subsection{设计原则}
\begin{enumerate}
\item \textbf{可重复}\\
 每个步骤中断后均可重复执行,不影响结果的一致性。某个步骤的执行不修改其他步骤的结果。
\item \textbf{依赖独立性}\\
原则上应不依赖于其他步骤的目录结构和执行结果,单个步骤可独立运行。
\item \textbf{参数独立性}\\
 在user.cfg中有一个以APP名命名的参数列表,除通用参数外不应调用其他APP的参数。
\item \textbf{模块化}\\
 一个APP就是一个模块,即一个Python脚本,添加新的APP不需要改动流程。
\item \textbf{可扩展}
\begin{itemize}
\item APP可由用户自定义,将其放置到相应目录即可像其他APP一样使用。
\item sample.list的解析方式也可由用户自定义,只需按固定的数据结构返回信息。
\end{itemize}



\end{enumerate}
\newpage
\section{流程框架}
\label{sec:2}
\subsection{流程框架图}
\begin{figure}[H]
%画流程图
\begin{tikzpicture}[outline/.style={draw=#1!50,thick,fill=#1!20}, outline/.default=black,outline=red]
\path (-5,5.2) node(tag1)  [module,fill=red!35,text width=5em] {}
(-2.5,5.2) node(tag1T)  [text width=7em] {Python模块}
(2,5.2) node(tag2)  [io,text width=0.25cm] {}
(5.5,5.2) node(tag2T)  [text width=9em] {InputOutput文件};
\path (-5.5,3) node(head1)  [head,fill=blue!50,text width=9em] {Input};
\path (-5.5,-1.3) node(bg-input)  [bg,text width=9em] {};
\path (-5.5,0.5) node(sample) [io] {sample.list}
(-5.5,-1.2) node(config) [io] {user.cfg};
%(-5.5,-2.9) node(config) [io] {APP.config};

\path (0.5,3) node(head2)  [head] {Main Module};
\path (0.5,-1.3) node(bg-mod)  [bg] {};
\path (0.4,1) node(parseSampleList) [module,text width=12em,fill=red!40] {Gaea.py}
(0.3,-0.4) node(parseSampleList) [module] {parseSampleList}
(0.3,-1.8) node(parseConfig) [module] {parseConfig}
(0.3,-3.2) node(parseApp) [module] {parseApp};

%APPs
\path (-5.5,-8) node(APPs)  [bgbox] {};
\path (-5.5,-7.2) node(filter) [app] {filter}
(-5.5,-8.2) node(alignment) [app] {alignment}
(-5.5,-9.2) node(...) [app] {...}
(-5.5,-6.2) node(apps) [] {APPs};

\draw[very thick,blue,->] (APPs)-- (bg-mod);



\path (6.5,3) node(head3)  [head,fill=blue!50,text width=9em] {Output};
\path (6.5,-1.3) node(bg-output)  [bg,text width=9em] {};
\path (6.5,0) node(dag) [io] {state.json}
(6.5,-2) node(workdir) [io] {scripts};

\draw[very thick,blue,->] (bg-input)-- (bg-mod);
\draw[very thick,blue,->] (bg-mod)--(bg-output);

\path (1,-7) node(head4)  [head,text width=17em] {任务提交监控};
\path (1,-10.7) node(bg-jobM)  [bg,text width=17em,minimum height=15em] {};
\path (1,-9.7) node(jobS) [module,text width=10em,minimum height=5em,fill=red!40] {jobSubmit.py};
\path (1,-11.8) node(jobM) [module,text width=10em,minimum height=5em,fill=red!40] {gaeaJobMonitor.py};

%\draw[very thick,blue,->] (dag.east)..controls (10,-1) and (8,-7)..node[below,sloped] {根据DAG描述,提交并监控任务}(jobM.east);
\draw[very thick,blue,->] (dag.east)..controls (10,-1) and (8,-6)..(jobS.east);

\end{tikzpicture}
\caption{GaeaPipeline流程框架图}
\end{figure}


%%%任务提交与监控
\subsection{任务提交与监控}
\label{sec:5}
jobSubmit.py是根据state.json信息提交任务的主程序,对每个样本的hadoop任务会重新生成gaea.sh脚本,然后和Standalone任务的shell脚本统一提交调度。其中gaea.sh中通过调用gaeaJobMonitor.py来执行和监控每个hadoop任务。
\begin{figure}[H]
\begin{tikzpicture}[outline/.style={draw=#1!50,thick,fill=#1!20}, outline/.default=black,outline=red]
\path 
(-1,0) node(jobS2) [module,text width=10em,minimum height=5em] {jobSubmit}
 (7,0) node(runG) [bgbox,text width=15em,minimum height=12em] {}
   (3,-4) node(ann) [text width=20em,minimum height=12em] {该模块将执行每个hadoop步骤的shell脚本,并捕获输出和状态信息,监控每个hadoop任务。}
 (7,1) node(word)[] {gaea.sh}
 (7,-0.8) node(jobM2) [module,text width=10em,minimum height=5em] {gaeaJobMonitor.py};
  \draw[very thick,->] (jobS2)--node[below,sloped] {生成gaea.sh}(runG);
  \draw[very thick,dashed,->] (jobM2)..controls (10,-1) and (9,-4) ..node[below,sloped] {注}(ann.east);
\end{tikzpicture}
\end{figure}


\section{自定义APP模块}
\label{sec:5}
\subsection{简述}
GaeaPipeline的一个APP模块就是一个按约定书写的python脚本,将写好的APP放到{\kaishu \hyperref[envset]{\$GAEA\_HOME}}/workflow目录下或在user.cfg中指定存放APP的路径即可使用。使用时只需在user.cfg中设置``analysis\_flow''即可{\kaishu \hyperref[fig:flow]{(如图)}}。

\subsection{书写约定}
\subsubsection{App的输入输出}
 APP的输入输出信息会存储到一个的字典类型的数据结构中\footnote{即程序执行后生成的state.json文件},APP解析时也需根据依赖关系,从该数据结构中查找所依赖步骤的输出信息,作为输入。
 
 \subsubsection{模块命名}
以一个名为HelloWorld的APP为例。按其运行平台,应将该模块命名为``S\_HelloWorld.py''或``H\_HelloWorld.py''。``S''指单机程序,``H''指Hadoop程序。
\subsubsection{主要方法与数据结构}

\newpage
\begin{enumerate}
\item \textbf{必要引用和继承}\\
\lstinputlisting[language=Python]{source/import.py}
\item \textbf{INIT属性}\\
INIT是定义默认参数的属性,其值将被存放到self中,会被user.cfg参数覆盖,用户要在user.cfg文件中设置的参数,需在INIT中设置一个默认值或空值,一些不常更改的参数在此设置也比较方便,如单机APP在此设置INIT.APP.mem(内存资源)等。

\item \textbf{参数调用}\\
user.cfg,INIT默认参数和以及继承自Workflow的信息,均会以字典的形式存储到self对象中,APP可对该对象中的数值进行修改,但除非必要不要在此修改其他APP的参数。程序完成后,会在workdir/scripts/state目录下生成一个state.json文件\footnote{提交任务后也会生成results.json文件,该文件主要信息是state.json的一个子集,但若提交了单机任务,此处会保存jobId信息},保存了所有self对象中的信息。

\item \textbf{impl类}\\
impl是一个传入的类,包含了一些开发APP的常用方法。
\begin{itemize}
\item impl.expath()是扩展程序路径的方法,如果传入的程序路径不是绝对路径,则会根据名称在相应目录中查找并返回绝对路径。该方法有两个参数,第一个参数 固定为self.Path.prgDir\footnote{由user.cfg设置的存放APP所用程序的目录,多个路径由":"分开,按顺序查找。},第二个参数为程序名,程序名一般存储形式为self.APP.prgParamName。
\item impl.mkdir() 是创建目录的方法,可有多个参数,返回值为所创建目录的绝对路径。原则上一个APP不能依赖其他APP创建的路径,所以要使用该方法,而不使用os.path.join().
\item impl.write\_shell()  该方法用于创建脚本。
\end{itemize}

\end{enumerate}



\newpage
\subsection{代码示例}
{\small {\kaishu 模块S\_HelloWorld.py代码:}}
\label{helloworld}
\lstinputlisting[language=Python]{source/S_HelloWorld.py}


\section{样本信息}
%\subsection{解析模块}
%样本信息

%\subsection{样本信息配置}
\begin{enumerate}
\item \textbf{华大标准下机数据}\\
华大下机数据文件中有详细的建库等信息,所以只需提供部分信息,每列信息以制表符隔开,每列信息见表8。Gaea 会自动生成 RG,并且在 fq1 和 fq2 所在目录寻找 adapter 序列文件。

\begin{table}[htp]
{\small
\begin{center}
\caption{华大标准下机数据配置表}
\begin{tabular}{|c|c|c|c|c|c|c|c|c|}
\hline
列号& 1         & 2      & 3      &4     & 5                &6                 &7                &8  \\
\hline
名称 & 样品名 & 性别 & 家系 & 类型 &建库 pool 号 &数据基本路径&上机 pool 号 &样品的子文库号(index)\\
\hline
\end{tabular}
\end{center}
\label{default}
 }
\end{table}%

\item \textbf{独立数据}\\
一个样品往往包含多个 lane 信息,其中每个 lane 需要配置如下格式,其中需要配置的属性包括:样品名,fq1,fq2,rg(reads group);可选择配置属性为:gender(性别),family(家系), pool,type(类型),libname(文库号),adp1,adp2。
\lstinputlisting[language=bash]{source/mode2.list}

\item \textbf{BAM数据}\\
如果已做完比对相关分析并得到 BAM 文件,可以用 Gaea 继续变异检测和注释。此时,只需要提供以制表符分隔的两列信息,两列信息分别是:样品名称和 BAM 文件路径或 BAM 文件所在文件夹的路径。需要注意的是,BAM 文件所在文件夹的路径中只能有 BAM 文件。

\item \textbf{VCF数据}\\
如果已经得到 VCF 文件,只需做后续分析,可以只配置 VCF 文件。此时,只需要提供以制表符分隔的两列信息,两列信息分别是:样品名称和 VCF 文件路径。
\end{enumerate}
%DAG
\newpage
\section{参数配置}
\subsection{用户参数配置文件格式}
用户参数配置文件(user.cfg)使用cfg格式(json格式不再使用)。
cfg文件一种是基于INI文件的配置文件,用``\#''号作为注释标记(不能用分号``;''),用中括号标记小节(section),可进行小节的嵌套,区分key值大小写,且支持多种数据类型,详见{\kaishu \href{http://configobj.readthedocs.org/en/latest/configobj.html#the-config-file-format}{格式说明}}。示例如下
\lstinputlisting[language=bash]{source/example.cfg}

\subsection{extend配置文件}
extend配置文件\footnote{该文件位于\$GAEA\_HOME/config/目录下, 如深圳集群: /ifs4/ISDC\_BD/GaeaProject/software/GaeaPipeline/config/extend.cfg}中提供了扩展user.cfg与具体应用无关的通用参数hadoop和ref的完整信息,目的在于简化用户配置,用户不需要再指定详细的集群参数等。简化后用户只需如下指定所使用集群:
\lstinputlisting[language=bash]{source/extend_hadoop.cfg}

user.cfg有较高的优先级,用户也可以在user.cfg的hadoop小节下指定完整的信息,这样则会覆盖extend中的设置{\kaishu \hyperref[C]{{\footnotesize 见附录2}}}。

%\newpage
\subsection{主要参数项说明}
用户需要提供的参数配置\footnote{用户配置文件相关example位于\$GAEA\_HOME/config/目录下},主要有以下几个要素:
\begin{itemize}
\item \textbf{analysis\_flow}\\
设置分析步骤和依赖关系,详见{\kaishu \hyperref[subs5.3]{下一小节}}。
\item \textbf{file}\\
需要经常更换,或多个APP通用的文件(如bed文件),需在此设置。另外,此参数项下的参数可被其它APP参数以一定的格式引用,见{\kaishu \hyperref[B]{附录1}}中genotype等APP的参数。
\item \textbf{hadoop}\\
Hadoop集群信息。该参数在流程的extend配置文件中有相应信息的映射,用户只需简单指定所用集群,文件系统等信息。
\item \textbf{ref}\\
参考序列信息。其具体路径同样保存在extend配置文件中,如果用户设置的不是绝对路径,将通过关键字在extend.cfg文件查找。
\item \textbf{Path}\\
指定APP程序的搜索路径和APP模块的额外存放路径。
\item \textbf{APP参数}\\
存放各个APP的程序和参数,参数项名称为各个APP名。其中有两个功能较特殊的APP:
\begin{itemize}
\item  init \\
init用于从接收sample信息到启动正式分析流程的过度,必要时会对数据进行解压,需在此参数项中设置qualitysystem等。此步骤始终是用户设置的起始步骤的依赖步骤,即事实上的起始步骤。
\item  self\_defined \\
用户可使用此参数项自定义APP。这也是一个名为self\_defined的APP模块,通过接收参数生成脚本,但对于较复杂的步骤建议直接实现APP模块。
\item  应用与参数要对应 \\
在已有的配置文件中修改时,若改变程序,要注意调整对应的参数。如: {\kaishu\textbf{alignment在使用aln和mem两个子工具时参数是不同的,不能只修改工具名而不改参数}}。
\end{itemize}

\end{itemize}


\subsection{步骤依赖关系}
\label{subs5.3}
流程中每个步骤的依赖关系是一个有向无环图,其示意图如下:%{\kaishu \hyperref[B]{如下}}。

\begin{figure}[H]
 \label{fig:APPdep}
\begin{tikzpicture}[fill=blue!20]
\path (-4,0) node (left-bound) []{};
\path (1,1.5) node(markblue) [circle,draw,fill,minimum size=2em] {}
(2.2,1.5) node (markblueT) {Hadoop}
(5,1.5) node(markred) [circle,draw,fill=red!20,minimum size=2em] {}
(6.3,1.5) node (markredT) {Standalone};
\path (-1,0) node(a) [circle,draw,fill] {A}
(0.5,0) node(b) [circle,draw,fill] {B}
(2,0) node(c) [circle,draw,fill] {C}
(4,0) node(e) [circle,draw,fill] {E}
(5,-1.5) node(g) [circle,draw,fill=red!20] {G}
(3,-1.5) node(d) [circle,draw,fill=red!20] {D}
(7,-3) node(h) [circle,draw,fill=red!20] {H}
(2,-3) node(f) [circle,draw,fill=red!20] {F};
  \draw[very thick,blue,->] (a) -- (b);
  \draw[very thick,blue,->] (b) -- (c);
  \draw[thick,red,->] (c) -- (d);
  \draw[thick,red,->] (b) -- (f);
  \draw[thick,red,->] (d) -- (e);
  \draw[thick,red,->] (e) -- (g);
  \draw[thick,red,->] (f) -- (h);
  \draw[thick,red,->] (g) -- (h);
 %\draw[help lines,dashed] (-5,-4) grid (5,4);
\end{tikzpicture}
  \caption{APP依赖关系示意图}
\end{figure}


流程的分析步骤需要在用户配置文件中设置。{\kaishu \hyperref[fig:flow]{下图}}是配置文件中``analysis\_flow"参数的设置形式,其中只有第一行的起始步骤没有依赖。另外,对执行步骤只依赖于前一步骤,且属于同一平台的步骤列表可以合并, 步骤间用``|''分割,代表前一步骤的输出是后一步骤的输入。

\begin{figure}[H]
 \label{fig:flow}
\begin{tikzpicture}
\tikzstyle{markbox}=[draw=red,text width=0.5em, minimum height=1em] 
\path (-3.9,0) node (left-bound) []{\lstinputlisting{source/workflow.cfg}};
\path (-4.7,0) node(boxmark0) [markbox,text width=2.1em] {}
(-3.6,0.) node(boxmark1) [markbox] {}
(-3.2,0) node(boxmark2) [markbox] {}
(-2.8,0) node(boxmark3) [markbox] {}
(3.2,1.9) node(ann0) [text width=25em,minimum height=2em] {analysis\_flow小节下的key,无实际意义,只需确保key不重复} 
(3.2,1) node(ann1) [text width=25em,minimum height=2em] {脚本运行平台} 
(3.2,0.5) node(ann2) [text width=25em,minimum height=2em] {运行的步骤(APP)}
(3.2,-1) node(ann3) [text width=25em,minimum height=3em] {运行步骤所依赖的步骤(APP),依赖步骤的输出将做为运行步骤的输入,单机步骤(S)将依此设置提交任务的依赖关系。};
\draw[very thick,blue,->] (boxmark0)..controls (-3,1.8) and (-2,2) ..(ann0);
\draw[very thick,blue,->] (boxmark1)..controls (-3,1.3) and (-2,1) ..(ann1);
\draw[very thick,blue,->] (boxmark2)..controls (-2.7,0.7) and (-2,0.5) ..(ann2);
\draw[very thick,blue,->] (boxmark3)..controls (-1.7,0) and (-2,-0.4) ..(ann3);
%\draw[help lines,dashed] (-5,-5) grid (5,5);
\end{tikzpicture}
 \caption{analysis\_flow参数}
\end{figure}

\subsection{用参数自定义APP}
在用户配置文件中可以仅通过参数实现自定义APP,参数APP的定义需在self\_defined下,以嵌入的子小节名作为APP名,需定义的key有以下几项:
\begin{itemize}
\item mem\\
设置所需计算资源,若是hadoop程序则不需要此项
\item output\\
定义输出路径,有\$\{WORKDIR\}和\$\{SAMPLE\}两个变量可用,\$\{WORKDIR\}指工作路径,\$\{SAMPLE\}指样本名。关于该变量的值可参考APP代码示例。
\item command\\
command为生成脚本的命令模版,定义在两个三引号内。可用变量为file参数项中的所有参数,以及\$\{INPUT\}和\$\{OUTPUT\}。
另外,变量\$\{checkstatus\}是检查上个命令执行状态信息的语句,需单独成一行。
\end{itemize}


%\newpage
参数定义APP示例:
\lstinputlisting[language=bash]{source/self.cfg}



\section{使用说明}
%\subsection{环境变量设置}
%\label{envset}
%\lstinputlisting[language=bash]{source/envset.sh}

\subsection{程序路径及示例}

\begin{itemize}
\item 程序路径
\begin{description}
\item[ 天津] /THL4/home/bgi\_gaea/software/GaeaPipeline/RunGaea
\item[ 深圳] /ifs4/ISDC\_BD/GaeaProject/software/GaeaPipeline/RunGaea
\end{description}
\item 相关示例在深圳集群 /ifs4/ISDC\_BD/GaeaProject/software/GaeaPipeline/example 目录下
\end{itemize}

\newpage
\subsection{生成并运行脚本}
\label{sec:3}
\subsubsection{准备}


\subsubsection{Usage}
RunGaea是用shell对Gaea.py进行的包装,不用配置环境变量\footnote{若使用Gaea.py需要设置环境变量,具体设置见GaeaPipeline/config/envset.sh},其功能是生成各分析步骤的shell脚本。其中,WORKDIR, SAMPLELIST和CONFIG为必须指定的参数。help信息如下:
\lstinputlisting[language=bash,keywordstyle=\color{black}]{source/gaea_usage.txt}
\subsubsection{投递任务}
用RunGaea生成脚本后,需要到工作目录的scripts目录下,执行run.sh提交任务。
\begin{itemize}
\item ./run.sh\\
	本地运行hadoop的任务调度程序,如在hadoop的master节点直接执行run.sh。
\item ./run.sh submit\\
	将hadoop任务提交到gaea.q队列\footnote{该队列为深圳hadoop35(cluster35)集群专用},单机任务提交到参数指定的队列。
\end{itemize}


\newpage
\subsubsection{项目目录结构}
%\lstinputlisting[]{source/workdirTree.txt}
每个分析项目会形成较为固定的目录结构。scrpts目录下是流程生成的脚本和log等其它信息,其中gaea目录下是hadoop脚本,standalone目录下是单机脚本,每个脚本执行后产生的标准错误和标准输出也会存在于与脚本相同的位置。%state目录下是几个文件

\begin{figure}[H]
\begin{tikzpicture}[outline/.style={draw=#1!50,thick,fill=#1!20}, outline/.default=black,outline=red]
\path (-0.9,3) node(head1)  [head,text width=46em] {WORKDIR};
\path (-0.9,-1) node(bg-mod)  [bg,text width=46em,minimum height=15em] {};

 \path 
(-5,0) node(scripts) [dir] {scripts} 
(-2,0) node(data) [dir] {raw\_data}
(1,0) node(tmp) [dir] {temp}
(4,0) node(fq) [dir] {fq}
(-5,-2) node(alignment) [dir] {alignment} 
(-2,-2) node(variation/) [dir] {variation}
(1,-2) node(annotion) [dir] {annotion}
(4,-2) node(etc) [dir] {app...}
;
\path (-4.4,-5) node(head2)  [head,text width=25em] {scripts};
 \path (-4.4,-8.2) node(scriptsdir)  [bg,text width=25em,minimum height=12em] {}
 (-7.4,-7) node(gaea) [dir] {gaea}
 (-4.4,-7) node(standalone) [dir] {standalone}
 (-1.4,-7) node(state) [dir] {state}
 (-6.4,-9) node(run) [dir,fill=red!30] {run.sh}
  (-3.4,-9) node(log) [dir,fill=red!15] {log};
%\draw[help lines,dashed] (-5,-4) grid (5,4);  

\path (4.2,-5) node(head3)  [head,text width=18em] {state};
 \path (4.2,-8.2) node(statedir)  [bg,text width=18em,minimum height=12em] {}
 (2.5,-7) node(statejson) [dir,fill=red!15] {state.json}
  (5.5,-7) node(results) [dir,fill=red!15] {results.json}
  (2.5,-9) node(rerun) [dir,fill=red!30] {rerun.list}
   (5.5,-9) node(fail) [dir,fill=red!15] {failed/success};
%\draw[help lines,dashed] (-5,-8) grid (5,8);  

 \draw[very thick,->] (scripts)..controls (-9,-3) and (-8,-4.5)..node[below,sloped] {}(scriptsdir);
  \draw[very thick,->] (state)..controls (0,-8)..node[below,sloped] {}(statedir);
\end{tikzpicture}
\caption{项目目录结构图}
\end{figure}

\subsection{查看错误信息并重跑}
任务执行结束后,用户需要查看scripts目录下的log文件(如有错误在state目录下会生成failed文件),如果有失败任务需要通过查看相应步骤的标准错误输出文件找到原因,例如发现是参数配置错误,则需修正后重新生成脚本,然后编辑scripts/state目录下的rerun.list,删掉不需执行的任务,然后提交任务即可。
\begin{itemize}
\item 
rerun.list文件,是一个包含样本名和运行步骤两列信息的文本文件,单样分析时能区分不同样本的步骤。程序将根据此文件提交任务,若要重跑只需删除不需要跑的步骤,然后重新提交即可。其格式如下:
\lstinputlisting[]{source/rurun.txt}
\end{itemize}


%\appendix
\newpage                                                                                                                                                                          
\begin{appendices}
\renewcommand\appendixpagename{附录}
\renewcommand\appendixtocname{附录}
\appendixpage
\renewcommand\thesection{\Roman{section}}
\renewcommand{\appendixname}{Appendix}

\section{用户参数配置文件}
 \label{B}
 {\color{cyan} \kaishu{深圳集群:}}
\lstinputlisting[language=Python]{source/user.cfg}

\newpage
\section{extend配置文件}
  \label{C}
  {\color{cyan} \kaishu{深圳集群:}}
  \lstinputlisting[language=bash]{source/extend.cfg}

\end{appendices}


\end{document}
